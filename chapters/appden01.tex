%% 附录页

\chapter{附录A\quad 科技写作中非学术型低级错误的主要表现}

本附录主要针对学位论文写作或中文科技论文写作,供重庆邮电大学学位论文查非工作参考。未尽事宜,可参考重庆邮电大学论文写作要求、重庆邮电大学学报编辑部等国内期刊社、出版社的通用出版规定。

推荐阅读《科学出版社作者编辑手册》、《科学道德与学风建设宣传参考大纲(试用本)》等写作指导性书籍或资料,可了解更多、更详尽的通用写作出版规范。

\textcolor{red}{(注:对于一些不宜放入正文中,但作为毕业设计(论文)又不可缺的组成部分或具有重要参考价值的内容,可编入毕业设计(论文)的附录中,例如,公式的推演、源程序代码、附图等内容。附录的内容为备选项目,作者可根据内容的需要决定附录的项目数,用附录A、附录B方式编号。附录的篇幅不宜太多。附录与主体部分一起编制页码。若附录部分有手工制作或复印件,手工制作或复印件部分要装订在内但可以不计页码。附录的文字按照正文格式进行排版。)}

\textcolor{red}{(附:科技写作中非学术型低级错误的主要表现见 moban文件夹下)}

\textbf{代码效果演示:}
\begin{python}
# 这是一段代码测试
def CQUPT():
    print("Hello,CQUPT")
    print("学长学长,给咱们讲讲3G芯片的故事呗。")
    print("这是世界上第一枚0.13微米工艺的TD-SCDMA 3G手机基带芯片。"
         +"它的诞生,标志着我国3G通信核心芯片等关键...")
CQUPT()
\end{python}
\newpage\quad %加一页,使得附录B第一页的页眉为 附录B

